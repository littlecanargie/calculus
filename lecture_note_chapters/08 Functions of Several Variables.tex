In the previous chapters, we only considered uni-variable functions, i.e. functions that takes one number as input.  However, in practice, the quantities we are interested in may be related to serveral variables.  For example, suppose we would like to know our net income $I$ in manufacturing and selling a type of product.  The net income would depend on several variables, including the unit cost for manufacturing the product ($C$), the unit price of the product ($P$), and the quantity of product sold ($Q$).  Therefore, we have 
\[I = (P-C)Q := f(P,C,Q)\]
Here, $f$ would then be a function that takes three numbers as input. 

Previously when we are aiming to visualize uni-variable function, 