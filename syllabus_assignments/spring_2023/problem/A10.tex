\providecommand{\pgfsyspdfmark}[3]{}
\providecommand{\savepicturepage}[3]{}

\documentclass[11pt,letterpaper]{article}
\usepackage[lmargin=1in,rmargin=1in,tmargin=1in,bmargin=1in]{geometry}

% -------------------
% Packages
% -------------------
\usepackage{
	amsmath,			% Math Environments
	amssymb,			% Extended Symbols
	enumerate,		    % Enumerate Environments
	graphicx,			% Include Images
	lastpage,			% Reference Lastpage
	multicol,			% Use Multi-columns
	multirow,			% Use Multi-rows
	gensymb
}


% -------------------
% Font
% -------------------
\usepackage[T1]{fontenc}
\usepackage{charter}


% -------------------
% Commands
% -------------------

\newcommand{\prob}{\noindent\textbf{Problem. }}
\newcounter{problem}
\newcommand{\problem}{
	\stepcounter{problem}%
	\noindent \textbf{Problem \theproblem. }%
}
\newcommand{\pspace}{\par\vspace{\baselineskip}}
\newcommand{\ds}{\displaystyle}


% -------------------
% Header & Footer
% -------------------
\usepackage{fancyhdr}

\fancypagestyle{pages}{
	%Headers
	\fancyhead[L]{}
	\fancyhead[C]{}
	\fancyhead[R]{}
\renewcommand{\headrulewidth}{0pt}
	%Footers
	\fancyfoot[L]{}
	\fancyfoot[C]{}
	\fancyfoot[R]{}
\renewcommand{\footrulewidth}{0.0pt}
}
\headheight=0pt
\footskip=14pt

\pagestyle{pages}


% -------------------
% Content
% -------------------
\begin{document}
\noindent\textbf{\large Calculus II (MSF\_10010 / AM\_\_1080AH) \\ 2023 Spring \\ Problem List X (Due May 11)}

\bigskip

\problem A solid is formed by revolving the area enclosed by $y = 2\sqrt{x}$, the $x$-axis and $x = 15$ around the $x$-axis.  Find the volume and surface area (including its round base perpendicular to the $x$-axis) of this solid. 
\vspace{15mm}

\problem In class, we derived the volume of a ball of radius $r$ as $\frac{4}{3}\pi r^3$ with the method of disks.  Use the method of shells to find the same volume.
\vspace{15mm}

\problem A solid is formed by revolving the area enclosed by $y = \frac{1}{x}$, $y = 2$, the $x$- and $y$-axes and $x = 2$ around the $y$-axis.  Find the volume of this solid. 

\smallskip
\noindent (Hint: You'll have to split the solid into two parts and calculate their volumes separately.)
\vspace{15mm}

\problem The \textit{Gabriel's horn} is constructed by revolving the curve of $y = \frac{1}{x}$ where $x \ge 1$ around the $x$-axis.  Show that the
volume of this solid of revolution is $\pi$.

\smallskip
\noindent (Hint: You may treat the upper limit of the $x$-coordinate for this solid as $\infty$ and evaluate the volume with an improper integral.)

\end{document}