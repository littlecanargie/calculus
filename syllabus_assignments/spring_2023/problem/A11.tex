\providecommand{\pgfsyspdfmark}[3]{}
\providecommand{\savepicturepage}[3]{}

\documentclass[11pt,letterpaper]{article}
\usepackage[lmargin=1in,rmargin=1in,tmargin=1in,bmargin=1in]{geometry}

% -------------------
% Packages
% -------------------
\usepackage{
	amsmath,			% Math Environments
	amssymb,			% Extended Symbols
	enumerate,		    % Enumerate Environments
	graphicx,			% Include Images
	lastpage,			% Reference Lastpage
	multicol,			% Use Multi-columns
	multirow,			% Use Multi-rows
	gensymb
}


% -------------------
% Font
% -------------------
\usepackage[T1]{fontenc}
\usepackage{charter}


% -------------------
% Commands
% -------------------

\newcommand{\prob}{\noindent\textbf{Problem. }}
\newcounter{problem}
\newcommand{\problem}{
	\stepcounter{problem}%
	\noindent \textbf{Problem \theproblem. }%
}
\newcommand{\pspace}{\par\vspace{\baselineskip}}
\newcommand{\ds}{\displaystyle}


% -------------------
% Header & Footer
% -------------------
\usepackage{fancyhdr}

\fancypagestyle{pages}{
	%Headers
	\fancyhead[L]{}
	\fancyhead[C]{}
	\fancyhead[R]{}
\renewcommand{\headrulewidth}{0pt}
	%Footers
	\fancyfoot[L]{}
	\fancyfoot[C]{}
	\fancyfoot[R]{}
\renewcommand{\footrulewidth}{0.0pt}
}
\headheight=0pt
\footskip=14pt

\pagestyle{pages}


% -------------------
% Content
% -------------------
\begin{document}
\noindent\textbf{\large Calculus II (MSF\_10010 / AM\_\_1080AH) \\ 2023 Spring \\ Problem List XI (Due May 18)}

\bigskip

\problem Evaluate the following integrals (note that they may not exist!):
\begin{enumerate}[a)]
    \item $\int_0^{\infty} \frac{1}{(x+1)^2}~dx$
    \item $\int_0^{\infty} x^2e^{-2x}~dx$
    \item $\int_{-27}^8 \frac{1}{(\sqrt[3]{x})^2}~dx$
    \item $\int_0^{\pi} \tan x~dx$
    \item $\int_0^\infty \sin x~dx$
\end{enumerate}
\vspace{3mm}

\problem A random variable $X$ is said to follow a uniform distribution of support $[a, b]$ if it has the following probability density function:
\[f_X(x) = \begin{cases}
    c, \quad a \le x \le b\\
    0, \quad \text{otherwise}
\end{cases}\]
\begin{enumerate}[a)]
    \item Express $c$ with $a$ and $b$ so that $f_X(x)$ is a probability density function.
    \item Derive the expected value of $X$.
    \item Derive the variance of $X$.
\end{enumerate}
\vspace{3mm}

\problem Although it seems intuitive that every random variable should have an expected value, this is not the case for some weird random variables.  Suppose a continuous random variable $X$ follows a standard Cauchy distribution and has the following probability density function
\[f_X(x) = \frac{1}{\pi(1+x^2)}\]
\begin{enumerate}[a)]
    \item Verify that $f_X(x)$ is a probability density function.
    \item Show that the expected value of $X$ does not exist.
\end{enumerate}
\vspace{3mm}

\problem Normal distribution is the most important and commonly used distribution for continuous variables.  The probability density function for a continuous variable distributed as a standard normal distribution is given by
\[\phi(x) = \frac{1}{\sqrt{2\pi}} e^{-x^2/2}\]
Given the fact that 
\[\int_{-\infty}^{\infty} e^{-t^2}~dt = \sqrt{\pi}\]
\begin{enumerate}[a)]
    \item Verify that $\phi(x)$ is a probability density function.
    \item Verify that the expected value of a standard normal distibution is $0$.  (Note: You also have to argue that the expected value exists!)
    \item Verify that the variance of a standard normal distribution is $1$. (Hint: Factor out one $x$ from the integrand and use integration by parts.)
\end{enumerate}

\end{document}