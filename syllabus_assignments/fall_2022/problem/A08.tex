\providecommand{\pgfsyspdfmark}[3]{}

\documentclass[11pt,letterpaper]{article}
\usepackage[lmargin=1in,rmargin=1in,tmargin=1in,bmargin=1in]{geometry}

% -------------------
% Packages
% -------------------
\usepackage{
	amsmath,			% Math Environments
	amssymb,			% Extended Symbols
	enumerate,		    % Enumerate Environments
	graphicx,			% Include Images
	lastpage,			% Reference Lastpage
	multicol,			% Use Multi-columns
	multirow,			% Use Multi-rows
	gensymb
}


% -------------------
% Font
% -------------------
\usepackage[T1]{fontenc}
\usepackage{charter}


% -------------------
% Commands
% -------------------

\newcommand{\prob}{\noindent\textbf{Problem. }}
\newcounter{problem}
\newcommand{\problem}{
	\stepcounter{problem}%
	\noindent \textbf{Problem \theproblem. }%
}
\newcommand{\answer}{\noindent \textbf{Answer. }}
\newcommand{\pspace}{\par\vspace{\baselineskip}}
\newcommand{\ds}{\displaystyle}


% -------------------
% Header & Footer
% -------------------
\usepackage{fancyhdr}

\fancypagestyle{pages}{
	%Headers
	\fancyhead[L]{}
	\fancyhead[C]{}
	\fancyhead[R]{}
\renewcommand{\headrulewidth}{0pt}
	%Footers
	\fancyfoot[L]{}
	\fancyfoot[C]{}
	\fancyfoot[R]{}
\renewcommand{\footrulewidth}{0.0pt}
}
\headheight=0pt
\footskip=14pt

\pagestyle{pages}


% -------------------
% Content
% -------------------
\begin{document}
\noindent\textbf{\large Calculus I (AM\_\_1050AH / MSF\_10110) \\ 2022 Fall \\ Differentiation rules II, III}

\bigskip

\problem Find the following limits
\begin{enumerate}[(a)]
    \item $\lim\limits_{x \to \infty} \big(1-\frac{2}{x}\big)^{3x}$
    \item $\lim\limits_{x \to 2} \big(\frac{2x}{x+2}\big)^{\frac{4}{x-2}}$
    \item $\lim\limits_{x \to \infty} (1+2^{-x})^{\left(2^x\right)}$
    \item $\lim\limits_{x \to 0} (1+2^{-x})^{\left(2^x\right)}$
    \item $\lim\limits_{x \to 0} (\cos x)^{\frac{2}{x^2}}$ \quad (Hint: Take the 2 in the exponent into the base)
\end{enumerate}\vspace{4mm}


\problem Find the derivatives of the following functions
\begin{enumerate}[(a)]
    \item $(x+1)^{2022}$
    \item $\ln(3x+4)$
    \item $\sqrt[3]{(x^2+4)^2}$
    \item $e^{-\frac{x}{8\pi}}$
    \item $\sin(x^2+1)$
\end{enumerate}\vspace{4mm}

\problem Find the derivatives of the following functions
\begin{enumerate}[(a)]
    \item $e^{\frac{2x+1}{x+3}}$
    \item $\tan^{-1}(\sqrt{x^2-1})$
    \item $\sqrt[4]{\sin\sqrt[3]{x}}$
    \item $\log_{(x^2+1)} 3$
    \item $x^x$ \quad (Hint: Write $x$ as $e^{\ln x}$)
\end{enumerate}\vspace{4mm}

\problem Find the following limits. You \textit{may} use the L'Hôpital's rule \textit{if applicable}.
\begin{enumerate}[(a)]
    \item $\lim\limits_{x \to 1} \frac{x^3+x^2+x-3}{x^3+2x^2+x-3}$
    \item $\lim\limits_{x \to 0} \frac{e^{(3x^2+2x)}-1}{\sin(2x^2+3x)}$
    \item $\lim\limits_{x \to 0} \frac{\sin (x^2)}{x \tan x}$
    \item $\lim\limits_{x \to 0} x^2 \ln (x^2)$ \quad (Hint: Transform it into $\frac{\infty}{\infty}$ form)
    \item $\lim\limits_{x \to 0} \frac{e^{-\frac{1}{x^2}}}{x^2}$ \quad (Hint: $\frac{0}{0}$ form can also be transformed into $\frac{\infty}{\infty}$ form)
\end{enumerate}\vspace{4mm}

% \problem Determine if the following statements are true or false and explain. (You can just provide a counterexample if you determine them as false)
% \begin{enumerate}[(a)]
%     \item If $f'(x) = g'(x)$ (for all $x\in \mathbb{R}$), then $f(x) = g(x)$
%     \item If $f(1) = 0$, then $f'(1) = 0$
%     \item If $f'(x) = 0$ (for all $x\in \mathbb{R}$), then $f(x) = 0$
% \end{enumerate}\vspace{6mm}

% \problem Let $f(x) = \sqrt[4]{x} - \sqrt{x}$,
% \begin{enumerate}[(a)]
%     \item Find the tangent line of $f(x)$ at the point where $x=16$.
%     \item At which point(s) on $f(x)$ is its tangent line horizontal?
%     \item Is $f(x)$ differentiable at $x = 0$? Why?
% \end{enumerate}\vspace{6mm}

% \problem A ball is expanding with its radius $r$ as a function of time $t$: $r(t) = \sqrt{t} + 2, t \ge 0$
% \begin{enumerate}[(a)]
%     \item Find the rate its radius is growing at $t = 1$
%     \item Find the rate its surface area is growing at $t = 1$
%     \item Find the rate its volume is growing at $t = 1$
% \end{enumerate}\vspace{6mm}

\end{document}