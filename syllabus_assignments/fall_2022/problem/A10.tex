\providecommand{\pgfsyspdfmark}[3]{}

\documentclass[11pt,letterpaper]{article}
\usepackage[lmargin=1in,rmargin=1in,tmargin=1in,bmargin=1in]{geometry}

% -------------------
% Packages
% -------------------
\usepackage{
	amsmath,			% Math Environments
	amssymb,			% Extended Symbols
	enumerate,		    % Enumerate Environments
	graphicx,			% Include Images
	lastpage,			% Reference Lastpage
	multicol,			% Use Multi-columns
	multirow,			% Use Multi-rows
	gensymb
}


% -------------------
% Font
% -------------------
\usepackage[T1]{fontenc}
\usepackage{charter}
\usepackage{xcolor}

% -------------------
% Commands
% -------------------

\newcommand{\prob}{\noindent\textbf{Problem. }}
\newcounter{problem}
\newcommand{\problem}{
	\stepcounter{problem}%
	\noindent \textbf{Problem \theproblem. }%
}
\newcommand{\answer}{\noindent \textbf{Answer. }}
\newcommand{\pspace}{\par\vspace{\baselineskip}}
\newcommand{\ds}{\displaystyle}


% -------------------
% Header & Footer
% -------------------
\usepackage{fancyhdr}

\fancypagestyle{pages}{
	%Headers
	\fancyhead[L]{}
	\fancyhead[C]{}
	\fancyhead[R]{}
\renewcommand{\headrulewidth}{0pt}
	%Footers
	\fancyfoot[L]{}
	\fancyfoot[C]{}
	\fancyfoot[R]{}
\renewcommand{\footrulewidth}{0.0pt}
}
\headheight=0pt
\footskip=14pt

\pagestyle{pages}


% -------------------
% Content
% -------------------
\begin{document}
\noindent\textbf{\large Calculus I (AM\_\_1050AH / MSF\_10110) \\ 2022 Fall \\ Application of Derivatives (5.1)}

\bigskip

\problem Let $f(x) = (2+x)^5e^{1+x}\sin(\pi x)$,
\begin{enumerate}[(a)]
    \item Give the best linear approximant for $f(x)$ near $x = 1$
    \item Give the best linear approximant for $f(x)$ near $x = 0$.
    \item Give the best linear approximant for $f(x)$ near $x = -1$.
\end{enumerate}\vspace{6mm}

\problem Let $f(x) = \tan^{-1} x$.
\begin{enumerate}[(a)]
    \item Give the best linear approximant for $f(x)$ near $x = 1$.
    \item Give the best quadratic approximant for $f(x)$ near $x = 1$.
    \item From (a) and (b), approximate $\tan^{-1}1.05$ using linear and quadratic approximation.
    \item Use a calculator to calculate $\tan^{-1} 1.05$.  What is the percentage of error for the linear and quadratic approximants (percentage of error = $\frac{\text{approximated value - true value}}{\text{true value}} \times 100\%$)?
\end{enumerate}\vspace{6mm}

\problem Use linear approximation to estimate $\sqrt[3]{7.98}$.  Give the percentage of error for your linear approximation. \vspace{6mm}

\problem Newton's law of gravity dictates that gravitational acceleration is governed by the inverse square law.  If you are $r$ meters away from a planet's center of mass, then the gravitational acceleration you feel from the planet can be expressed as a function of $r$:
\[g(r) = \frac{GM}{r^2}\]
where $G$ is the gravitational constant (unit: $m^3/(kg \cdot s^2)$) and $M$ is the mass of that planet (unit: $kg$).  Therefore, the farther you are from the planet, the smaller gravitational tug you will feel from the planet.  Suppose the earth is a perfect sphere with radius $R = 6.4 \times 10^{6}$ meters and mass $M_e$ kilograms, while its center of mass is also its spherical center.
\begin{enumerate}[(a)]
    \item Express the gravitational acceleration at the surface of the earth with $G$, $M_e$ and $R$.
    \item Denote the height of a mountain as $kR$.  Express the gravitational acceleration at the peak of that mountain with $G$, $M_e$, $R$ and $k$.
    \item Since the height of every mountain is minuscule compared to the radius of the earth, $k$ is a very small positive number.  Give a linear approximation to the result of (b) around $k = 0$.
    \item From (c), suppose the gravitational acceleration at the surface of the earth is $9.8 m/s^2$, then what would be the gravitational acceleration at the peak of a $3200$-meter mountain?
\end{enumerate}
\vspace{6mm}

% \problem Use the hinted linear approximations to approximate the following quantities:
% \begin{enumerate}[(a)]
%     \item $\tan 46 \degree$, approximating $\tan x$ at $x = 45 \degree$. (Note: You'll have to operate in radians) 
%     \item $\ln(1.01)$, approximating $\ln (1+x)$ at $x = 0$.
%     \item $\tan^{-1}0.99$, approximating $\tan^{-1} x$ at $x = 1$.
%     \item $\sqrt[4]{80}$, approximating $3\sqrt[4]{1+x}$ at $x = 0$.
%     \item $\frac{1}{0.99^3}$, approximating $\frac{1}{(1+x)^3}$ at $x = 0$.
% \end{enumerate}\vspace{6mm}

% \problem Find the following limits. You \textit{may} use the L'Hôpital's rule \textit{if applicable}.
% \begin{enumerate}[(a)]
%     \item $\lim\limits_{x \to 1} \frac{x^3+x^2+x-3}{x^3+2x^2+x-3}$
%     \item $\lim\limits_{x \to 0} \frac{e^{(3x^2+2x)}-1}{\sin(2x^2+3x)}$
%     \item $\lim\limits_{x \to 0} \frac{\sin (x^2)}{x \tan x}$
%     \item $\lim\limits_{x \to 0} x^2 \ln (x^2)$ \quad (Hint: Transform it into $\frac{\infty}{\infty}$ form)
%     \item $\lim\limits_{x \to 0} \frac{e^{-\frac{1}{x^2}}}{x^2}$ \quad (Hint: $\frac{0}{0}$ form can also be transformed into $\frac{\infty}{\infty}$ form)
% \end{enumerate}\vspace{4mm}

% \problem Determine if the following statements are true or false and explain. (You can just provide a counterexample if you determine them as false)
% \begin{enumerate}[(a)]
%     \item If $f'(x) = g'(x)$ (for all $x\in \mathbb{R}$), then $f(x) = g(x)$
%     \item If $f(1) = 0$, then $f'(1) = 0$
%     \item If $f'(x) = 0$ (for all $x\in \mathbb{R}$), then $f(x) = 0$
% \end{enumerate}\vspace{6mm}

% \problem Let $f(x) = \sqrt[4]{x} - \sqrt{x}$,
% \begin{enumerate}[(a)]
%     \item Find the tangent line of $f(x)$ at the point where $x=16$.
%     \item At which point(s) on $f(x)$ is its tangent line horizontal?
%     \item Is $f(x)$ differentiable at $x = 0$? Why?
% \end{enumerate}\vspace{6mm}

% \problem A ball is expanding with its radius $r$ as a function of time $t$: $r(t) = \sqrt{t} + 2, t \ge 0$
% \begin{enumerate}[(a)]
%     \item Find the rate its radius is growing at $t = 1$
%     \item Find the rate its surface area is growing at $t = 1$
%     \item Find the rate its volume is growing at $t = 1$
% \end{enumerate}\vspace{6mm}

\end{document}