\providecommand{\pgfsyspdfmark}[3]{}

\documentclass[11pt,letterpaper]{article}
\usepackage[lmargin=1in,rmargin=1in,tmargin=1in,bmargin=1in]{geometry}

% -------------------
% Packages
% -------------------
\usepackage{
	amsmath,			% Math Environments
	amssymb,			% Extended Symbols
	enumerate,		    % Enumerate Environments
	graphicx,			% Include Images
	lastpage,			% Reference Lastpage
	multicol,			% Use Multi-columns
	multirow,			% Use Multi-rows
	gensymb
}


% -------------------
% Font
% -------------------
\usepackage[T1]{fontenc}
\usepackage{charter}


% -------------------
% Commands
% -------------------

\newcommand{\prob}{\noindent\textbf{Problem. }}
\newcounter{problem}
\newcommand{\problem}{
	\stepcounter{problem}%
	\noindent \textbf{Problem \theproblem. }%
}
\newcommand{\pspace}{\par\vspace{\baselineskip}}
\newcommand{\ds}{\displaystyle}


% -------------------
% Header & Footer
% -------------------
\usepackage{fancyhdr}

\fancypagestyle{pages}{
	%Headers
	\fancyhead[L]{}
	\fancyhead[C]{}
	\fancyhead[R]{}
\renewcommand{\headrulewidth}{0pt}
	%Footers
	\fancyfoot[L]{}
	\fancyfoot[C]{}
	\fancyfoot[R]{}
\renewcommand{\footrulewidth}{0.0pt}
}
\headheight=0pt
\footskip=14pt

\pagestyle{pages}


% -------------------
% Content
% -------------------
\begin{document}
\noindent\textbf{\large Calculus I (AM\_\_1050AH / MSF\_10110) \\ 2022 Fall \\ Homework List for Pre-calculus I}

\bigskip

\problem A parallelogram is a quadrilateral with opposing sides parallel to each other.  Some of its geometrical properties are: (1) Opposing sides have the same length (2) The midpoint of diagonals coincide (3) The sum of square of all it sides is equal to the sum of square of its diagonals. Give four points $A (1, 0), B (4, 6), C (5, 3), D (2, -3)$,
\begin{enumerate}[(a)]
    \item Graph the points on a Cartesian plane
    \item Show that $ABCD$ is a parallelogram by checking if its opposing sides are parallel
    \item Check property (1) by demonstrating $\overline{AB} = \overline{CD}$, $\overline{BC} = \overline{AD}$
    \item Check property (2) by demonstrating the coordinates of the midpoints for $\overline{AC}$ and $\overline{BD}$
    \item Check property (3) by demonstrating $\overline{AB}^2 + \overline{BC}^2 + \overline{CD}^2 + \overline{DA}^2 = \overline{AC}^2 + \overline{BD}^2$
\end{enumerate} \vspace{6mm}

\problem Given two points $A$ and $B$ with coordinates $(2, 12)$ and $(1, 9)$ on an $x$-$y$ plane, and line $L$ passing through both $A$ and $B$,
	\begin{enumerate}[(a)]
	\item Give an expression for $L$
	\item Find the $x$-intercept and $y$-intercept of $L$.
	\item Let $L_1$ be $L$ shifted to the right for 3 units then up for 2 units. Given and expression for $L_1$.
	\item Let $L_2$ be a line \textit{parallel} to $L$ that passes through the origin. Give an expression for $L_2$.
	\item Let $L_3$ be a line \textit{perpendicular} to $L$ that passes through $A$. Give an expression for $L_3$.
	\end{enumerate} \vspace{6mm}

\problem A curve on a Cartesian plane has equation $(x-2)^2+(y-3)^2 = 10$,
	\begin{enumerate}[(a)]
	\item For any point on this curve, what is the distance between the point and point $O(2, 3)$?
	\item Based on the above, what is the shape of this curve?
	\item Graph the curve. Does this curve represent a function of $x$, why?
	\item Where does (do) this curve intersect with the line $y = x + 1$?
	\end{enumerate} \vspace{6mm}

\problem Graph the function $f(x) = \frac{1}{\sqrt{x-1}} + 3$ and give its domain, range and inverse function \vspace{6mm}

\end{document}