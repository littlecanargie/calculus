\providecommand{\pgfsyspdfmark}[3]{}

\documentclass[11pt,letterpaper]{article}
\usepackage[lmargin=1in,rmargin=1in,tmargin=1in,bmargin=1in]{geometry}

% -------------------
% Packages
% -------------------
\usepackage{
	amsmath,			% Math Environments
	amssymb,			% Extended Symbols
	enumerate,		    % Enumerate Environments
	graphicx,			% Include Images
	lastpage,			% Reference Lastpage
	multicol,			% Use Multi-columns
	multirow,			% Use Multi-rows
	gensymb
}


% -------------------
% Font
% -------------------
\usepackage[T1]{fontenc}
\usepackage{charter}


% -------------------
% Commands
% -------------------

\newcommand{\prob}{\noindent\textbf{Problem. }}
\newcounter{problem}
\newcommand{\problem}{
	\stepcounter{problem}%
	\noindent \textbf{Problem \theproblem. }%
}
\newcommand{\answer}{\noindent \textbf{Answer. }}
\newcommand{\pspace}{\par\vspace{\baselineskip}}
\newcommand{\ds}{\displaystyle}


% -------------------
% Header & Footer
% -------------------
\usepackage{fancyhdr}

\fancypagestyle{pages}{
	%Headers
	\fancyhead[L]{}
	\fancyhead[C]{}
	\fancyhead[R]{}
\renewcommand{\headrulewidth}{0pt}
	%Footers
	\fancyfoot[L]{}
	\fancyfoot[C]{}
	\fancyfoot[R]{}
\renewcommand{\footrulewidth}{0.0pt}
}
\headheight=0pt
\footskip=14pt

\pagestyle{pages}


% -------------------
% Content
% -------------------
\begin{document}
\noindent\textbf{\large Calculus I (AM\_\_1050AH / MSF\_10110) \\ 2022 Fall \\ Differentiation rules I, II}

\bigskip

\problem Find the derivatives of the following functions
\begin{enumerate}[(a)]
    \item $(x^3+1) \sin x$
    \item $x^{-2} \ln x$
    \item $\frac{x-1}{(x+2)^2}$
    \item $e^x \sec x$
    \item $(\sin 2x) (\cot x)$
\end{enumerate}\vspace{6mm}

\problem Find the derivatives of the following functions
\begin{enumerate}[(a)]
    \item $(e^x + \ln 2)^2$
    \item $\sin^3 x$
    \item $\ln \frac{\pi}{\sqrt[3]{x^2}}$
    \item $3^{\ln x}$ \quad (Hint: Write $3$ as $e^{\ln 3}$)
    \item $\log_{x^5} 5^x$ \quad (Hint: Try changing the base)
\end{enumerate}\vspace{6mm}


\problem A heap of radioactive material is decaying so that its mass (in grams) over time (in thousand years) follows the equation $M(t) = 50 (0.9715)^t, \; t \ge 0$
\begin{enumerate}[(a)]
    \item What is its mass at $t = 0$?
    \item What is the half-life of this material, i.e. the time needed to halve its mass?
    \item At what rate is this heap of material decaying initially (at $t=0$)?
    \item At what rate is this heap of material decaying at its half-life?
\end{enumerate}\vspace{6mm}

\problem A raindrop is falling with its speed (in meters/second) over time (in seconds) as $v(t) = 5\cdot \frac{e^{4t}-1}{e^{4t}+1}, t \ge 0$
\begin{enumerate}[(a)]
    \item What is the raindrop's initial velocity, i.e. its velocity at $t = 0$?
    \item What is the raindrop's terminal velocity, i.e. its velocity as $t \rightarrow \infty$?
    \item The acceleration function $a(t)$ is defined as the derivative of the velocity function with respect to time. Find $a(t)$.
    \item What is the raindrop's initial acceleration?
    \item What is the raindrop's terminal acceleration?
\end{enumerate}\vspace{6mm}

% \problem Determine if the following statements are true or false and explain. (You can just provide a counterexample if you determine them as false)
% \begin{enumerate}[(a)]
%     \item If $f'(x) = g'(x)$ (for all $x\in \mathbb{R}$), then $f(x) = g(x)$
%     \item If $f(1) = 0$, then $f'(1) = 0$
%     \item If $f'(x) = 0$ (for all $x\in \mathbb{R}$), then $f(x) = 0$
% \end{enumerate}\vspace{6mm}

% \problem Let $f(x) = \sqrt[4]{x} - \sqrt{x}$,
% \begin{enumerate}[(a)]
%     \item Find the tangent line of $f(x)$ at the point where $x=16$.
%     \item At which point(s) on $f(x)$ is its tangent line horizontal?
%     \item Is $f(x)$ differentiable at $x = 0$? Why?
% \end{enumerate}\vspace{6mm}

% \problem A ball is expanding with its radius $r$ as a function of time $t$: $r(t) = \sqrt{t} + 2, t \ge 0$
% \begin{enumerate}[(a)]
%     \item Find the rate its radius is growing at $t = 1$
%     \item Find the rate its surface area is growing at $t = 1$
%     \item Find the rate its volume is growing at $t = 1$
% \end{enumerate}\vspace{6mm}

\end{document}