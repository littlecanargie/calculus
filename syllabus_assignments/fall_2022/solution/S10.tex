\providecommand{\pgfsyspdfmark}[3]{}

\documentclass[11pt,letterpaper]{article}
\usepackage[lmargin=1in,rmargin=1in,tmargin=1in,bmargin=1in]{geometry}

% -------------------
% Packages
% -------------------
\usepackage{
	amsmath,			% Math Environments
	amssymb,			% Extended Symbols
	enumerate,		    % Enumerate Environments
	graphicx,			% Include Images
	lastpage,			% Reference Lastpage
	multicol,			% Use Multi-columns
	multirow,			% Use Multi-rows
	gensymb
}


% -------------------
% Font
% -------------------
\usepackage[T1]{fontenc}
\usepackage{charter}
\usepackage{xcolor}

% -------------------
% Commands
% -------------------

\newcommand{\prob}{\noindent\textbf{Problem. }}
\newcounter{problem}
\newcommand{\problem}{
	\stepcounter{problem}%
	\noindent \textbf{Problem \theproblem. }%
}
\newcommand{\answer}{\noindent \textbf{Answer. }}
\newcommand{\pspace}{\par\vspace{\baselineskip}}
\newcommand{\ds}{\displaystyle}


% -------------------
% Header & Footer
% -------------------
\usepackage{fancyhdr}

\fancypagestyle{pages}{
	%Headers
	\fancyhead[L]{}
	\fancyhead[C]{}
	\fancyhead[R]{}
\renewcommand{\headrulewidth}{0pt}
	%Footers
	\fancyfoot[L]{}
	\fancyfoot[C]{}
	\fancyfoot[R]{}
\renewcommand{\footrulewidth}{0.0pt}
}
\headheight=0pt
\footskip=14pt

\pagestyle{pages}


% -------------------
% Content
% -------------------
\begin{document}
\noindent\textbf{\large Calculus I (AM\_\_1050AH / MSF\_10110) \\ 2022 Fall \\ Application of Derivatives (5.1) (Answer Key)}

\bigskip

\problem Let $f(x) = (2+x)^5e^{1+x}\sin(\pi x)$,
\begin{enumerate}[(a)]
    \item Give the best linear approximant for $f(x)$ near $x = 1$
    \item Give the best linear approximant for $f(x)$ near $x = 0$.
    \item Give the best linear approximant for $f(x)$ near $x = -1$.
\end{enumerate}\vspace{6mm}

\answer 

\noindent To obtain the best linear approximant, we first find the derivative of $f$:
\begin{align*}
    f'(x) &= [(2+x)^5]'e^{1+x}\sin(\pi x)+(2+x)^5][e^{1+x}]'\sin(\pi x)+(2+x)^5e^{1+x}[\sin(\pi x)]'\\
    &= 5(2+x)^4e^{1+x}\sin(\pi x) + (2+x)^5e^{1+x}\sin(\pi x) + (2+x)^5e^{1+x}\pi\cos(\pi x)\\
    &=[(2+x)^4e^{1+x}][(7+x)\sin(\pi x)+\pi(2+x)\cos(\pi x)]
\end{align*}
\begin{enumerate}[(a)]
    \item $f(1)+f'(1)(x-1) = 0 + [3^4e^2][\pi \cdot 3 \cdot (-1)](x-1) = -243e^2\pi(x-1)$
    \item $f(0)+f'(0)(x-0) = 0 + [2^4e][\pi \cdot 2 \cdot 1]x = 32e\pi x$
    \item $f(-1)+f'(-1)(x-(-1)) = 0 + [1^4e^0][\pi \cdot 1 \cdot (-1)](x+1) = -\pi(x+1)$
\end{enumerate}
\noindent Alternatively, we can use approximations of individual functions:
\begin{enumerate}[(a)]
    \item Let $u = x-1$, then $u$ is close to $0$ as $x$ is close to $1$, and we have
    \begin{align*}
        f(x) &= (3+u)^5e^{2+u}\sin(\pi+\pi u) \\
        &= 3^5(1+u/3)^5e^2e^u[-\sin(\pi u)] \\
        &\approx -243e^2(1+5u/3)(1+u)(\pi u) \\
        &\approx -243e^2 \pi u = -243e^2\pi(x-1)
    \end{align*}
    The second approximation discards all terms with $u^2$ or higher.
    \item In this approximation, $x$ is already close to $0$, so we operate directly in $x$:
    \begin{align*}
        f(x) &= (2+x)^5e^{1+x}\sin(\pi x) \\
        &= 2^5(1+x/2)^5 e \cdot e^x \sin(\pi x) \\
        &\approx 32e(1+5x/2)(1+x)(\pi x) \\
        &\approx 32e \pi x
    \end{align*}
    \item Let $u = x+1$, then $u$ is close to $0$ as $x$ is close to $-1$, and we have
    \begin{align*}
        f(x) &= (1+u)^5e^{u}\sin(\pi u-\pi) \\
        &= (1+u)^5e^u[-\sin(\pi u)]\\
        &\approx (1+5u)(1+u)(-\pi u) \\
        &\approx -\pi u = -\pi(x+1)
    \end{align*}
    The second approximation discards all terms with $u^2$ or higher.
\end{enumerate}

\vspace{6mm}

\problem Let $f(x) = \tan^{-1} x$.
\begin{enumerate}[(a)]
    \item Give the best linear approximant for $f(x)$ near $x = 1$.
    \item Give the best quadratic approximant for $f(x)$ near $x = 1$.
    \item From (a) and (b), approximate $\tan^{-1}1.05$ using linear and quadratic approximation.
    \item Use a calculator to calculate $\tan^{-1} 1.05$.  What is the percentage of error for the linear and quadratic approximants (percentage of error = $\frac{\text{approximated value - true value}}{\text{true value}} \times 100\%$)?
\end{enumerate}\vspace{6mm}

\answer To obtain the best linear and quadratic approximant, we first obtain the first and second derivatives of $f(x)$:
\[f'(x) = \frac{1}{1+x^2} \qquad f''(x) = \frac{d}{dx} \Big(\frac{1}{\textcolor{blue}{1+x^2}}\Big)= -\frac{1}{(\textcolor{blue}{1+x^2})^2}(\textcolor{blue}{2x}) = -\frac{2x}{(1+x^2)^2}\]
where the first derivative comes from the handout on derivatives of inverse functions. 
\begin{enumerate}[(a)]
    \item The best linear approximant at $x=1$ is 
    \[ f(1) + f'(1)(x-1) = \tan^{-1}{1} + \frac{1}{1+1^2}(x-1) = \frac{\pi}{4} + \frac{1}{2}(x-1)\]
    \item The best linear approximant at $x=1$ is 
    \[f(1) + f'(1)(x-1) + \frac{1}{2}f''(1)(x-1)^2 = \frac{\pi}{4} + \frac{1}{2}(x-1) + \frac{1}{2}\Big(-\frac{2 \cdot 1}{(1+1^2)^2}\Big)(x-1)^2 = \frac{\pi}{4} + \frac{1}{2}(x-1) - \frac{1}{4}(x-1)^2\]
    \item Linear approximation: $\frac{\pi}{4} + \frac{1}{2}\cdot 0.05 = \frac{\pi}{4} + 0.025 \approx 0.81040$; Quadratic approximation: $\frac{\pi}{4} + \frac{1}{2}\cdot 0.05 - \frac{1}{4} \cdot 0.05^2 = \frac{\pi}{4} + 0.024375 \approx 0.80977$
    \item $\tan^{-1} 1.05 \approx 0.80978$.  Therefore, the percentage of error for the linear approximate is about $\frac{0.81040-0.80978}{0.80978} \approx 0.0766\%$; the percentage of error for the quadratic approximate is about $\frac{0.80977-0.80978}{0.80978} \approx -0.0012\%$
\end{enumerate}\vspace{6mm}

\problem Use linear approximation to estimate $\sqrt[3]{7.98}$.  Give the percentage of error for your linear approximation. \vspace{6mm}

\answer We consider the function $f(x) = \sqrt[3]{x}$, then it is trivial that $f(8) = \sqrt[3]{8} = 2$.  Since $7.98$ is very close to $8$, we can use linear approximation of $f(x)$ at $x=8$ to approximate $\sqrt[3]{7.98} = f(7.98)$.  The linear approximant is given by:
\[f(8) + f'(8)(x-8) = 2 + \Big(\frac{d}{dx}\sqrt[3]{x}\Big)\Big|_{x=8}(x-8) = 2 + \Big(\frac{1}{3\sqrt[3]{x^2}}\Big)\Big|_{x=8}(x-8) = 2 + \frac{x-8}{12}\]
Therefore, our linear approximate is $2 + \frac{7.98-8}{12} \approx 1.99833$, and its percentage of error is $\frac{1.9983333-\sqrt[3]{8}}{\sqrt[3]{8}} \approx \frac{1.9983333-1.9983319}{1.9983319} \approx 7 \times 10^{-5}\%$
\vspace{6mm}

\problem Newton's law of gravity dictates that gravitational acceleration is governed by the inverse square law.  If you are $r$ meters away from a planet's center of mass, then the gravitational acceleration you feel from the planet can be expressed as a function of $r$:
\[g(r) = \frac{GM}{r^2}\]
where $G$ is the gravitational constant (unit: $m^3/(kg \cdot s^2)$) and $M$ is the mass of that planet (unit: $kg$).  Therefore, the farther you are from the planet, the smaller gravitational tug you will feel from the planet.  Suppose the earth is a perfect sphere with radius $R = 6.4 \times 10^{6}$ meters and mass $M_e$ kilograms, while its center of mass is also its spherical center.
\begin{enumerate}[(a)]
    \item Express the gravitational acceleration at the surface of the earth with $G$, $M_e$ and $R$.
    \item Denote the height of a mountain as $kR$.  Express the gravitational acceleration at the peak of that mountain with $G$, $M_e$, $R$ and $k$.
    \item Since the height of every mountain is minuscule compared to the radius of the earth, $k$ is a very small positive number.  Give a linear approximation to the result of (b) around $k = 0$.
    \item From (c), suppose the gravitational acceleration at the surface of the earth is $9.8 m/s^2$, then what would be the gravitational acceleration at the peak of a $3200$-meter mountain?
\end{enumerate}
\vspace{6mm}

\answer 
\begin{enumerate}[(a)]
    \item Since the surface of the earth is $R$ meters from the spherical center of the earth, the gravitational acceleration there is $\frac{GM_e}{R^2}$ (m/s$^2$).
    \item Since the mountain is $kR$ meters in height, the peak of the mountain is $R+kR$ meters from the spherical center of the earth.  Therefore, the gravitational acceleration there is $\frac{GM_e}{(R+kR)^2} = \frac{1}{(1+k)^2}\frac{GM_e}{R^2}$ (m/s$^2$).
    \item $\frac{1}{(1+k)^2}\frac{GM_e}{R^2}= (1+k)^{-2}\frac{GM_e}{R^2} \approx (1-2k)\frac{GM_e}{R^2}$, where the approximation step is based on $k \approx 0$.
    \item From (a), since the gravitational acceleration at the surface of the earth is $9.8$ m/s$^2$, we have $\frac{GM_e}{R^2} = 9.8$.  Further, since the mountain is $3200$ meters high, if we denote the height of the mountain as $kR$, then we have $k = \frac{3200}{R} = \frac{3200}{6400000} = 0.0005$.  Therefore, from (c), the gravitational acceleration at the peak of the mountain would be approximately $(1-2k)\frac{GM_e}{R^2} = (1-0.001)9.8 = 9.8 - 0.0098 = 9.7902$ (m/s$^2$).
\end{enumerate}
\vspace{6mm}

% \problem Use the hinted linear approximations to approximate the following quantities:
% \begin{enumerate}[(a)]
%     \item $\tan 46 \degree$, approximating $\tan x$ at $x = 45 \degree$. (Note: You'll have to operate in radians) 
%     \item $\ln(1.01)$, approximating $\ln (1+x)$ at $x = 0$.
%     \item $\tan^{-1}0.99$, approximating $\tan^{-1} x$ at $x = 1$.
%     \item $\sqrt[4]{80}$, approximating $3\sqrt[4]{1+x}$ at $x = 0$.
%     \item $\frac{1}{0.99^3}$, approximating $\frac{1}{(1+x)^3}$ at $x = 0$.
% \end{enumerate}\vspace{6mm}

% \problem Find the following limits. You \textit{may} use the L'Hôpital's rule \textit{if applicable}.
% \begin{enumerate}[(a)]
%     \item $\lim\limits_{x \to 1} \frac{x^3+x^2+x-3}{x^3+2x^2+x-3}$
%     \item $\lim\limits_{x \to 0} \frac{e^{(3x^2+2x)}-1}{\sin(2x^2+3x)}$
%     \item $\lim\limits_{x \to 0} \frac{\sin (x^2)}{x \tan x}$
%     \item $\lim\limits_{x \to 0} x^2 \ln (x^2)$ \quad (Hint: Transform it into $\frac{\infty}{\infty}$ form)
%     \item $\lim\limits_{x \to 0} \frac{e^{-\frac{1}{x^2}}}{x^2}$ \quad (Hint: $\frac{0}{0}$ form can also be transformed into $\frac{\infty}{\infty}$ form)
% \end{enumerate}\vspace{4mm}

% \problem Determine if the following statements are true or false and explain. (You can just provide a counterexample if you determine them as false)
% \begin{enumerate}[(a)]
%     \item If $f'(x) = g'(x)$ (for all $x\in \mathbb{R}$), then $f(x) = g(x)$
%     \item If $f(1) = 0$, then $f'(1) = 0$
%     \item If $f'(x) = 0$ (for all $x\in \mathbb{R}$), then $f(x) = 0$
% \end{enumerate}\vspace{6mm}

% \problem Let $f(x) = \sqrt[4]{x} - \sqrt{x}$,
% \begin{enumerate}[(a)]
%     \item Find the tangent line of $f(x)$ at the point where $x=16$.
%     \item At which point(s) on $f(x)$ is its tangent line horizontal?
%     \item Is $f(x)$ differentiable at $x = 0$? Why?
% \end{enumerate}\vspace{6mm}

% \problem A ball is expanding with its radius $r$ as a function of time $t$: $r(t) = \sqrt{t} + 2, t \ge 0$
% \begin{enumerate}[(a)]
%     \item Find the rate its radius is growing at $t = 1$
%     \item Find the rate its surface area is growing at $t = 1$
%     \item Find the rate its volume is growing at $t = 1$
% \end{enumerate}\vspace{6mm}

\end{document}