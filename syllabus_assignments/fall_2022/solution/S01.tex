\providecommand{\pgfsyspdfmark}[3]{}

\documentclass[11pt,letterpaper]{article}
\usepackage[lmargin=1in,rmargin=1in,tmargin=1in,bmargin=1in]{geometry}

% -------------------
% Packages
% -------------------
\usepackage{
	amsmath,			% Math Environments
	amssymb,			% Extended Symbols
	enumerate,		    % Enumerate Environments
	graphicx,			% Include Images
	lastpage,			% Reference Lastpage
	multicol,			% Use Multi-columns
	multirow,			% Use Multi-rows
	gensymb
}


% -------------------
% Font
% -------------------
\usepackage[T1]{fontenc}
\usepackage{charter}


% -------------------
% Commands
% -------------------

\newcommand{\prob}{\noindent\textbf{Problem. }}
\newcounter{problem}
\newcommand{\problem}{
	\stepcounter{problem}%
	\noindent \textbf{Problem \theproblem. }%
}
\newcommand{\answer}{\noindent \textbf{Answer. }}
\newcommand{\pspace}{\par\vspace{\baselineskip}}
\newcommand{\ds}{\displaystyle}


% -------------------
% Header & Footer
% -------------------
\usepackage{fancyhdr}

\fancypagestyle{pages}{
	%Headers
	\fancyhead[L]{}
	\fancyhead[C]{}
	\fancyhead[R]{}
\renewcommand{\headrulewidth}{0pt}
	%Footers
	\fancyfoot[L]{}
	\fancyfoot[C]{}
	\fancyfoot[R]{}
\renewcommand{\footrulewidth}{0.0pt}
}
\headheight=0pt
\footskip=14pt

\pagestyle{pages}


% -------------------
% Content
% -------------------
\begin{document}
\noindent\textbf{\large Calculus I (AM\_\_1050AH / MSF\_10110) \\ 2022 Fall \\ Homework List for Pre-calculus I (Answer Key)}

\bigskip

\problem A parallelogram is a quadrilateral with opposing sides parallel to each other.  Some of its geometrical properties are: (1) Opposing sides have the same length (2) The midpoint of diagonals coincide (3) The sum of square of all it sides is equal to the sum of square of its diagonals. Give four points $A (1, 0), B (4, 6), C (5, 3), D (2, -3)$,
\begin{enumerate}[(a)]
    \item Graph the points on a Cartesian plane
    \item Show that $ABCD$ is a parallelogram by checking if its opposing sides are parallel
    \item Check property (1) by demonstrating $\overline{AB} = \overline{CD}$, $\overline{BC} = \overline{AD}$
    \item Check property (2) by demonstrating the coordinates of the midpoints for $\overline{AC}$ and $\overline{BD}$
    \item Check property (3) by demonstrating $\overline{AB}^2 + \overline{BC}^2 + \overline{CD}^2 + \overline{DA}^2 = \overline{AC}^2 + \overline{BD}^2$
\end{enumerate} \vspace{6mm}

\answer

\begin{enumerate}[(a)]
    \item (Deferred)
    \item Two segments are parallel if and only if their slopes are identical.  Therefore, if we write the slopes of $\overline{AB}$, $\overline{BC}$, $\overline{CD}$, $\overline{DA}$ as $m_{AB}, m_{BC}, m_{CD}, m_{DA}$, we can just show if $m_{AB} = m_{CD}$ and $m_{BC} = m_{DA}$, which are true since $m_{AB}=\frac{6-0}{4-1}=2=\frac{(-3)-3}{2-5}=m_{CD}$, $m_{BC}=\frac{3-6}{5-4}=-3=\frac{0-(-3)}{1-2}=m_{DA}$.
    \item $\overline{AB} = \sqrt{(4-1)^2+(6-0)^2} = 3\sqrt{5} = \sqrt{(2-5)^2+((-3)-3)^2} = \overline{CD}$; $\overline{BC} = \sqrt{(5-4)^2+(3-6)^2} = \sqrt{10} = \sqrt{(2-1)^2+((-3)-0)^2} = \overline{AD}$.
    \item The midpoint for $\overline{AC}$ is $\left(\frac{1+5}{2},\frac{0+3}{2}\right) = (3, 1.5)$, and the midpoint for $\overline{BD}$ is $\left(\frac{4+2}{2}, \frac{6+(-3)}{2}\right) = (3, 1.5)$, and they coincide.
    \item We have shown in (c) that $\overline{AB} = \overline{CD} = 3\sqrt{5}$, $\overline{BC} = \overline{DA} = \sqrt{10}$.  Therefore, $\overline{AB}^2 + \overline{BC}^2 + \overline{CD}^2 + \overline{DA}^2 = 110$.  Since $\overline{AC}^2 + \overline{BD}^2 = [(5-1)^2+(3-0)^2]+[(2-4)^2+((-3)-6)^2] = 110$, we have $\overline{AB}^2 + \overline{BC}^2 + \overline{CD}^2 + \overline{DA}^2 = \overline{AC}^2 + \overline{BD}^2$.
\end{enumerate} \vspace{6mm}


\problem Given two points $A$ and $B$ with coordinates $(2, 12)$ and $(1, 9)$ on an $x$-$y$ plane, and line $L$ passing through both $A$ and $B$,
	\begin{enumerate}[(a)]
	\item Give an expression for $L$
	\item Find the $x$-intercept and $y$-intercept of $L$.
	\item Let $L_1$ be $L$ shifted to the right for 3 units then up for 2 units. Given and expression for $L_1$.
	\item Let $L_2$ be a line \textit{parallel} to $L$ that passes through the origin. Give an expression for $L_2$.
	\item Let $L_3$ be a line \textit{perpendicular} to $L$ that passes through $A$. Give an expression for $L_3$.
	\end{enumerate} \vspace{6mm}
	
\answer

\begin{enumerate}[(a)]
    \item The slope of $L$ is $\frac{9-12}{1-2} = 3$ and $L$ passes through $A(2,12)$, so we can express $L$ as $y-12 = 3(x-2)$, i.e. $y=3x+6$.
    \item The $x$-intercept can be found by setting $y=0$ in $L: y=3x+6$, which leads to $x=-2$, so the $x$-intercept is $-2$;  The $y$-intercept can be found by setting $x=0$ in $L: y=3x+6$, which leads to $y=6$, so the $y$-intercept is $6$.
    \item $L_1$ can be expressed as $y-2 = 3(x-3)+6$, i.e. $y = 3x-1$.
    \item $L_2$ is parallel to $L$, so they have the same slope, $3$.  Accompanied with the fact that $L_2$ passes through the origin $(0,0)$, it can be expressed as $y-0 = 3(x-0)$, i.e. $y=3x$.
    \item $L_3$ is perpendicular to $L$, so their slopes multiplies to (-1), which means that the slope of $L_3$ is $\frac{-1}{3} = -\frac{1}{3}$.  Accompanied with the fact that $L_3$ passes through $A(2,12)$, it can be expressed as $y-12 = -\frac{1}{3}(x-2)$, i.e. $y=-\frac{1}{3}x+\frac{38}{3}$.
\end{enumerate} \vspace{6mm}

\problem A curve on a Cartesian plane has equation $(x-2)^2+(y-3)^2 = 10$,
	\begin{enumerate}[(a)]
	\item For any point on this curve, what is the distance between the point and point $O(2, 3)$?
	\item Based on the above, what is the shape of this curve?
	\item Graph the curve. Does this curve represent a function of $x$, why?
	\item Where does (do) this curve intersect with the line $y = x + 1$?
	\end{enumerate} \vspace{6mm}

\answer

\begin{enumerate}[(a)]
    \item For any point $A(x_0, y_0)$ on the curve, $\overline{AO} = \sqrt{(x_0-2)^2+(y_0-3)^2} = \sqrt{10}$, where the last equality holds since $A$ is on a curve with equation $(x-2)^2+(y-3)^2 = 10$.
    \item Since every point on the curve is equidistant to the point $O$, the curve should be a circle.
    \item The graph should show a circle with center $(2,3)$ and radius $\sqrt{10}$.  Since if you draw a horizontal line $y=3$, it crosses the curve at two points, this curve does \textit{not} represent a function of $x$.
    \item Plug in $y=x+1$ into the equation yields $(x-2)^2+(x-2)^2 = 10$, whose solution is $x=2 \pm \sqrt{10}$. Plugging these solution back in $y=x+1$ yields the two intersection points $(2+\sqrt{10}, 3+\sqrt{10})$ and $(2-\sqrt{10}, 3-\sqrt{10})$
\end{enumerate} \vspace{6mm}


\problem Graph the function $f(x) = \frac{1}{\sqrt{x-1}} + 3$ and give its domain, range and inverse function \vspace{6mm}

\answer The graph is deferred.  For the domain of the function, since the expression in the square root cannot be negative, and the expression in the denominator cannot be zero, we have $x-1 \ge 0$ \text{and} $\sqrt{x-1} \ne 0$.  Solving these equations gives us the domain $\{x: x > 1\}$, or $(1,\infty)$.  For the range, when $x$ is a number greater than $1$ but getting close to $1$, $\frac{1}{\sqrt{x-1}}$ is a positive number growing indefinitely, thus so is $f(x)$.  Alternatively, when $x$ gets larger and larger, $\frac{1}{\sqrt{x-1}}$ becomes a  positive number that gets closer and closer to $0$, or $f(x)$ becomes a number greater than but closer and closer to $3$.  Therefore, the range of $f(x)$ is $(3, \infty)$.  For the inverse function, we solve $y=f(x)=\frac{1}{\sqrt{x-1}}+3$ for $x$ and yield $x = \frac{1}{(y-3)^2}+1$.  Therefore, the inverse function for $f(x)$ is $y=\frac{1}{(x-3)^2}+1$ with domain $(3, \infty)$ (the range of $f(x)$).

\end{document}