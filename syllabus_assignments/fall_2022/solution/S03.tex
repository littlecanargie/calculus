\providecommand{\pgfsyspdfmark}[3]{}
\providecommand{\savepicturepage}[3]{}

\documentclass[11pt,letterpaper]{article}
\usepackage[lmargin=1in,rmargin=1in,tmargin=1in,bmargin=1in]{geometry}

% -------------------
% Packages
% -------------------
\usepackage{
	amsmath,			% Math Environments
	amssymb,			% Extended Symbols
	enumerate,		    % Enumerate Environments
	graphicx,			% Include Images
	lastpage,			% Reference Lastpage
	multicol,			% Use Multi-columns
	multirow,			% Use Multi-rows
	gensymb
}


% -------------------
% Font
% -------------------
\usepackage[T1]{fontenc}
\usepackage{charter}


% -------------------
% Commands
% -------------------

\newcommand{\prob}{\noindent\textbf{Problem. }}
\newcounter{problem}
\newcommand{\problem}{
	\stepcounter{problem}%
	\noindent \textbf{Problem \theproblem. }%
}
\newcommand{\answer}{\noindent \textbf{Answer. }}
\newcommand{\pspace}{\par\vspace{\baselineskip}}
\newcommand{\ds}{\displaystyle}


% -------------------
% Header & Footer
% -------------------
\usepackage{fancyhdr}

\fancypagestyle{pages}{
	%Headers
	\fancyhead[L]{}
	\fancyhead[C]{}
	\fancyhead[R]{}
\renewcommand{\headrulewidth}{0pt}
	%Footers
	\fancyfoot[L]{}
	\fancyfoot[C]{}
	\fancyfoot[R]{}
\renewcommand{\footrulewidth}{0.0pt}
}
\headheight=0pt
\footskip=14pt

\pagestyle{pages}


% -------------------
% Content
% -------------------
\begin{document}
\noindent\textbf{\large Calculus I (AM\_\_1050AH / MSF\_10110) \\ 2022 Fall \\ Homework List for Pre-calculus III / Limits and Continuity I (Answer Key)}

\bigskip

\problem Evaluate the following quantities:
\begin{enumerate}[(a)]
    \item $\sin 225\degree$
    \item $\cos (-330\degree)$
    \item $\tan\frac{7}{4}\pi$
    \item $\cot\frac{5}{3}\pi$
\end{enumerate}\vspace{6mm}

\answer

\begin{enumerate}[(a)]
    \item $\sin 225\degree = \sin (180+45)\degree = -\frac{1}{\sqrt{2}}$
    \item $\cos (-330\degree) = \cos (-330+360)\degree = \cos 30\degree = \frac{\sqrt{3}}{2}$
    \item $\tan\frac{7}{4}\pi = \tan \left(\frac{7}{4}\pi - 2\pi\right) = \tan \left(-\frac{\pi}{4}\right) = -1$
    \item $\cot\frac{5}{3}\pi = \frac{1}{\tan\frac{5}{3}\pi} = \frac{1}{\tan \left(\frac{5}{3}\pi - 2\pi\right)} = \frac{1}{\tan \left(-\frac{\pi}{3}\right)} = \frac{1}{-\sqrt{3}} = -\frac{1}{\sqrt{3}}$
\end{enumerate}\vspace{6mm}

\problem Let $u = \tan \theta, 3\pi < \theta < \frac{7}{2}\pi$. Express the following quantities with $u$: 
\begin{enumerate}[(a)]
    \item $\sec \theta$
    \item $\cos \theta$
    \item $\cos 2\theta$
    \item $\cos\left(\frac{\theta}{2}\right)$
\end{enumerate}\vspace{6mm}

\answer
\begin{enumerate}[(a)]
    \item We know $\sec^2\theta = 1+\tan^2(\theta) = 1+u^2$, so $\sec \theta = \pm \sqrt{1+u^2}$ and we need to determine its sign.  Since $3\pi < \theta < \frac{7}{2}\pi$, we have $\cos \theta < 0$, so $\sec \theta = \frac{1}{\cos \theta} <0$, therefore $\sec \theta = -\sqrt{1+u^2}$.
    \item $\cos \theta = \frac{1}{\sec \theta} = -\frac{1}{\sqrt{1+u^2}}$
    \item $\cos 2\theta = 2 \cos^2\theta - 1 = 2 \frac{1}{1+u^2} - 1 = \frac{1-u^2}{1+u^2}$
    \item $\cos\left(\frac{\theta}{2}\right) = \pm \sqrt{\frac{1+\cos \theta}{2}} = \pm \sqrt{\frac{1-\frac{1}{\sqrt{1+u^2}}}{2}}$ and we need to determine its sign.  Since $3\pi < \theta < \frac{7}{2}\pi$, we have $\frac{3}{2}\pi < \frac{\theta}{2} < \frac{7}{4}\pi$, so $\cos \left(\frac{\theta}{2}\right) > 0$.  Therefore, $\cos \left(\frac{\theta}{2}\right) = \sqrt{\frac{1-\frac{1}{\sqrt{1+u^2}}}{2}}$
\end{enumerate}\vspace{6mm}

\problem A triangle $ABC$ is placed on a Cartesian plane with coordinates $A(0, 0)$ and $B(c, 0)$ where $c>0$. Point $C$ is in the first or second quadrant where $\overline{AC} = b$ and $\angle BAC = \theta$, $0 < \theta < \pi$
\begin{enumerate}[(a)]
    \item Give the coordinate of $C$
    \item Show that the area of triangle $ABC$ is $\frac{1}{2} bc \sin \theta$
    \item Show that $\overline{BC}^2 = \overline{AC}^2 + \overline{AB}^2 - 2\overline{AC}~\overline{AB} \cos\theta$
\end{enumerate}\vspace{6mm}

\answer
\begin{enumerate}[(a)]
    \item Let the coordinate of $C$ be $(x_0, y_0)$, then $\overline{AC} = \sqrt{(x_0-0)^2+(y_0-0)^2} = b$.  Also, by the coordinate definition of sines and cosines, $\sin \theta = \frac{y_0}{\sqrt{x_0^2+y_0^2}}$ and $\cos \theta = \frac{x_0}{\sqrt{x_0^2+y_0^2}}$. Combining the above we yield $y_0 = b\sin \theta$ and $x_0 = b\cos \theta$, so the coordinate of $C$ is $(b\cos \theta, b\sin \theta)$
    \item Using $\overline{AB}$ as base, the height of triangle $ABC$ is the $y$-coordinate of $C$, since $C$ is in the first or second quadrant.  Therefore, the area of the triangle is $\frac{1}{2}(\overline{AB})(b\sin \theta) = \frac{1}{2}c(b \sin \theta) = \frac{1}{2} bc \sin \theta$
    \item $\overline{BC}^2 = (b \cos \theta - c)^2 + (b \sin \theta - 0)^2 = (b^2 \cos^2 \theta - 2bc \cos \theta + c^2) + b^2 \sin^2 \theta = b^2(\cos^2 \theta + \sin^2  \theta) + c^2 - 2 bc \cos \theta = b^2 + c^2 - 2bc \cos \theta = \overline{AC}^2 + \overline{AB}^2 - 2\overline{AC}~\overline{AB} \cos\theta$
\end{enumerate}\vspace{6mm}

\problem Let $f(x) = \frac{\sin 4x}{|\sin 2x|}$, 
\begin{enumerate}[(a)]
    \item Plot $f(x)$ on a Cartesian plane
    \item What is the domain, range and period of $f(x)$?
    \item Evaluate $f\left(\frac{\pi}{4}\right), \lim\limits_{x \to  \left(\frac{\pi}{4}\right)^-} f(x)$ and $\lim\limits_{x \to \left(\frac{\pi}{4}\right)^+} f(x)$
    \item Evaluate $f(\pi), \lim\limits_{x \to  \pi^-} f(x)$ and $\lim\limits_{x \to \pi^+} f(x)$
\end{enumerate}

\answer
\begin{enumerate}[(a)]
    \item Please refer to previous course videos for the graph.  We first note that $f(x) = \frac{2\sin 2x\cos 2x}{|\sin 2x|} = 2\cos 2x \frac{\sin 2x}{|\sin 2x|}$.  That is:
    \[f(x) = \left\{
    \begin{array}{lr}
        2\cos 2x, &  \sin 2x > 0\\
        \text{undefined}, & \sin 2x = 0\\
        -2\cos 2x, & \sin 2x < 0
    \end{array}
    \right.\]
    Since $\sin 2x$ is a sine function with period squeezed to $\frac{2\pi}{2} = \pi$, we have
    \[\sin 2x \left\{
        \begin{array}{ll}
            >0, & x \in \left(k\pi, \left(k+\frac{1}{2}\right)\pi\right)\\
            =0, & x = \frac{k}{2}\pi\\
            <0, & x \in \left(\left(k-\frac{1}{2}\right)\pi, k\pi\right)
        \end{array}
    \right., k \in \mathbb{Z}\]
    Therefore, to graph $f(x)$, we first graph $2\cos 2x$, which is a cosine function with double the amplitude ($2$) and half the period ($\pi$).  We then let the points where $x = \frac{k}{2}\pi, k \in \mathbb{Z}$ be hollow, i.e. $f(x)$ is undefined at those inputs. Notice that these hollow points occurs at where $2 \cos 2x = \pm 2$.  Finally, we flip the graph along the $x$-axis at intervals where $x \in \left(\left(k-\frac{1}{2}\right)\pi, k\pi\right), k \in \mathbb{Z}$ to reflect the negative sign.   
    \item The domain, as shown in (a), is real numbers except points where $x = \frac{k}{2}\pi, k\in\mathbb{Z}$, i.e. $\{x: x\ne \frac{k}{2}\pi, k \in \mathbb{Z}\}$.  The range of the function is $(-2,2)$, which is the range of $2 \cos 2x$ without $2$ and $-2$ since $f(x)$ is undefined at the points where $2 \cos 2x = \pm 2$.  The period is $\frac{\pi}{2}$, which is half the period of $2 \cos 2x$, as shown in the graph that the period halved after we flipped the function at certain intervals.
    \item From the graph, we have $f\left(\frac{\pi}{4}\right) = \lim\limits_{x \to  \left(\frac{\pi}{4}\right)^-} f(x) = \lim\limits_{x \to  \left(\frac{\pi}{4}\right)^+} f(x) = 0$
    \item From the graph, we have $f(\pi)$ is \textit{undefined}, $\lim\limits_{x \to  \pi^-} f(x) = -2$ and $\lim\limits_{x \to  \pi^+} f(x) = 2$
\end{enumerate}


% \problem Let $f(x) = \left\{\begin{array}{lr} 
%         \frac{x^2-1}{x-1}, & x < 1\\
%         2, & x = 1\\
%         \frac{3x^2+2x-1}{|x^2-4x+3|}, & 1 < x < 3\\
%         \log_{10}(x-3), & x > 3
%         \end{array}\right.$
% \begin{enumerate}[(a)]
%     \item Evaluate $f(1), \lim\limits_{x \to  1^-} f(x)$ and $\lim\limits_{x \to 1^+} f(x)$
%     \item Evaluate $\lim\limits_{x \to  1} f(x)$. Is $f(x)$ continuous at $x=1$?
%     \item Evaluate $f(3), \lim\limits_{x \to  3^-} f(x)$ and $\lim\limits_{x \to 3^+} f(x)$
%     \item Evaluate $\lim\limits_{x \to  3} f(x)$. Is $f(x)$ continuous at $x=3$?
% \end{enumerate}\vspace{6mm}

% \problem Evaluate the following limits:
% \begin{enumerate}[(a)]
%     \item $\lim\limits_{x \to 0^+} \frac{x^2+1}{x}$
%     \item $\lim\limits_{x \to 0} \frac{x}{\sqrt{x+3}-\sqrt{3}}$
%     \item $\lim\limits_{x \to 0} \frac{(1-x)^{-\frac{1}{2}}-1}{x}$
%     \item $\lim\limits_{x \to 0} \frac{\sin 5x}{x}$
% \end{enumerate}\vspace{6mm}

\end{document}