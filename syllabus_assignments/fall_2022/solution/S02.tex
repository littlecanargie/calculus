\providecommand{\pgfsyspdfmark}[3]{}

\documentclass[11pt,letterpaper]{article}
\usepackage[lmargin=1in,rmargin=1in,tmargin=1in,bmargin=1in]{geometry}

% -------------------
% Packages
% -------------------
\usepackage{
	amsmath,			% Math Environments
	amssymb,			% Extended Symbols
	enumerate,		    % Enumerate Environments
	graphicx,			% Include Images
	lastpage,			% Reference Lastpage
	multicol,			% Use Multi-columns
	multirow,			% Use Multi-rows
	gensymb
}


% -------------------
% Font
% -------------------
\usepackage[T1]{fontenc}
\usepackage{charter}


% -------------------
% Commands
% -------------------

\newcommand{\prob}{\noindent\textbf{Problem. }}
\newcounter{problem}
\newcommand{\problem}{
	\stepcounter{problem}%
	\noindent \textbf{Problem \theproblem. }%
}
\newcommand{\answer}{\noindent \textbf{Answer. }}
\newcommand{\pspace}{\par\vspace{\baselineskip}}
\newcommand{\ds}{\displaystyle}


% -------------------
% Header & Footer
% -------------------
\usepackage{fancyhdr}

\fancypagestyle{pages}{
	%Headers
	\fancyhead[L]{}
	\fancyhead[C]{}
	\fancyhead[R]{}
\renewcommand{\headrulewidth}{0pt}
	%Footers
	\fancyfoot[L]{}
	\fancyfoot[C]{}
	\fancyfoot[R]{}
\renewcommand{\footrulewidth}{0.0pt}
}
\headheight=0pt
\footskip=14pt

\pagestyle{pages}


% -------------------
% Content
% -------------------
\begin{document}
\noindent\textbf{\large Calculus I (AM\_\_1050AH / MSF\_10110) \\ 2022 Fall \\ Homework List for Pre-calculus I, II (Answer Key)}

\bigskip

\problem Let $f(x) = -x^2 + 3x + 18$. $f(x)$ is a parabola when graphed on a Cartesian plane.
    \begin{enumerate}[(a)]
    \item  What direction is the parabola facing, why?
    \item Evaluate $f(0)$
    \item Find the coordinate of the vertex of $f(x)$
    \item Find the roots of $f(x)$
    \item Based on the above, sketch $f(x)$ and mark the vertex and $x-$, $y-$intercepts. 
    \item What is the domain and range of $f(x)$?
    \end{enumerate} \vspace{6mm}

\answer
    \begin{enumerate}[(a)]
    \item The parabola is facing downwards since the coefficient of $x^2$ is less than zero.
    \item $f(0) = -0^2 + 3 \cdot 0 + 18 = 18$
    \item $f(x) = -x^2 + 3x + 18 = -(x^2-3x)+18 = -\left(x^2-2\cdot\frac{3}{2}x+\left(\frac{9}{4}\right)^2\right)+18+\left(\frac{9}{4}\right)^2 = -\left(x-\frac{3}{2}\right)^2 + \frac{81}{4}$.  Therefore, the vertex of $f(x)$ is $(\frac{3}{2}, \frac{81}{4})$
    \item In (c) we've shown $f(x) = -\left(x-\frac{3}{2}\right)^2 + \frac{81}{4}$.  Setting $f(x) = 0$ we yield $x-\frac{3}{2} \pm \sqrt{\frac{81}{4}}$, which leads to two roots $x=6$ and $x=-3$.
    \item (Deferred)
    \item Any real number can be plugged into $f(x)$, so its domain is $\mathbb{R}$.  For the range, since $f(x)$ is a down-facing parabola with the $y$-coordinate of its vertex as $\frac{81}{4}$, its value maxes at $\frac{81}{4}$ and does not have a minimum, so its range is $\left(-\infty, \frac{81}{4}\right)$
    \end{enumerate} \vspace{6mm}

\problem Let $f(x) = -x^4 + x^3 + x - 1$. We will now try to factorize $f(x)$:
	\begin{enumerate}[(a)]
	\item Is $x = 1$ a root of $f(x)$, why? 
	\item Derive $g(x) = \frac{f(x)}{x-1}$. Is $x = 1$ a root of $g(x)$, why?
	\item Derive $h(x) = \frac{g(x)}{x-1}$. Determine the number of real roots for $h(x)$ using its discriminant
	\item Factorize $f(x)$
	\end{enumerate} \vspace{6mm}
	
\answer
\begin{enumerate}[(a)]
	\item Yes, since $f(1) = -1^4+1^3+1-1 = 0$.
	\item Using long division, we yield $g(x) = -x^3+1$.  Since $g(1) = -1^3+1 = 0$, $x=1$ is still a root for $g(x)$.
	\item Using long division, we yield $h(x) = -x^2-x-1$, which is a quadratic (degree-$2$) polynomial.  Since the discriminant of $h(x)$ is $(-1)^2-4(-1)(-1) = -3 < 0$, $h(x)$ has $0$ real roots.
	\item From the above we found that $f(x) = (x-1)g(x) = (x-1)(x-1)h(x) = -(x-1)^2(x^2+x+1)$, and we have shown that we cannot factorize further since $h(x)$ does not have real roots.  Therefore, the factorization is complete.
\end{enumerate} \vspace{6mm}

\problem Let $\log_{10} x = u, \log_{10} y = v$, where $x, y > 0$. Express the following with $u, v$:
\begin{enumerate}[(a)]
\item $\log_{0.001}x$
\item $\log_{10}{\frac{\sqrt[5]{x}}{y^2}}$
\item $\log_{x}\sqrt{10}$
\item $\log_{x}y$
\end{enumerate} \vspace{6mm}

\answer
\begin{enumerate}[(a)]
\item $\log_{0.001}x = \log_{10^{-3}}x = \frac{\log_{10}x}{\log_{10}10^{-3}} = \frac{u}{-3} = -\frac{u}{3}$
\item $\log_{10}{\frac{\sqrt[5]{x}}{y^2}} = \log_{10}{\frac{x^{\frac{1}{5}}}{y^2}} =  \log_{10}x^{\frac{1}{5}} - \log_{10}y^2 = \frac{1}{5}\log_{10}x - 2\log_{10}y = \frac{u}{5}-2v$
\item $\log_{x}\sqrt{10} = \log_{x}10^{\frac{1}{2}} = \frac{\log_{10}10^{\frac{1}{2}}}{\log_{10}x} = \frac{\frac{1}{2}}{u} = \frac{1}{2u}$
\item $\log_{x}y = \frac{\log_{10}y}{\log_{10}x} = \frac{v}{u}$
\end{enumerate} \vspace{6mm}

\problem Solve the following equation:
\[\log_{\frac{1}{\sqrt[3]{10}}}x - 3\log_{x}100 - 9 = 0\]
(Hint: Change all the logarithms to base 10) \vspace{6mm}

\answer 

\begin{align*}
    \frac{\log_{10}x}{\log_{10}\frac{1}{\sqrt[3]{10}}} - 3\frac{\log_{10}100}{\log_{10}x} - 9 &= 0\\
    \frac{\log_{10}x}{-1/3} - 3\frac{2}{\log_{10}x} - 9 &= 0\\
    -3y - \frac{6}{y} - 9 &= 0 \quad\text{(Let }y = \log_{10}{x}\text{)}\\
    y^2 + 2 + 3y &= 0 \quad \text{(Multiply }-\frac{y}{3}\text{ on both sides)}\\
    (y+1)(y+2) &= 0 \; \Rightarrow \; y = -1 \text{ or } -2 \; \Rightarrow \; x = 10^{-1} \text{ or } 10^{-2}
\end{align*}
Note that since $y = 0$ is not in our solution, we did not create extra roots by multiplying $-y/3$ on both sides of the equation.


\end{document}